\documentclass[]{article}
\usepackage{amssymb,amsmath}
\usepackage{ifxetex,ifluatex}
\ifxetex
  \usepackage{fontspec,xltxtra,xunicode}
  \defaultfontfeatures{Mapping=tex-text,Scale=MatchLowercase}
\else
  \ifluatex
    \usepackage{fontspec}
    \defaultfontfeatures{Mapping=tex-text,Scale=MatchLowercase}
  \else
    \usepackage[utf8]{inputenc}
  \fi
\fi
\usepackage{ctable}
\usepackage{float} % provides the H option for float placement
\usepackage{graphicx}
% We will generate all images so they have a width \maxwidth. This means
% that they will get their normal width if they fit onto the page, but
% are scaled down if they would overflow the margins.
\makeatletter
\def\maxwidth{\ifdim\Gin@nat@width>\linewidth\linewidth
\else\Gin@nat@width\fi}
\makeatother
\let\Oldincludegraphics\includegraphics
\renewcommand{\includegraphics}[1]{\Oldincludegraphics[width=\maxwidth]{#1}}
\ifxetex
  \usepackage[setpagesize=false, % page size defined by xetex
              unicode=false, % unicode breaks when used with xetex
              xetex,
              colorlinks=true,
              linkcolor=blue]{hyperref}
\else
  \usepackage[unicode=true,
              colorlinks=true,
              linkcolor=blue]{hyperref}
\fi
\hypersetup{breaklinks=true, pdfborder={0 0 0}}
\setlength{\parindent}{0pt}
\setlength{\parskip}{6pt plus 2pt minus 1pt}
\setlength{\emergencystretch}{3em}  % prevent overfull lines
\setcounter{secnumdepth}{0}

\title{t-test Template}
\author{Rapport package team @ https://github.com/aL3xa/rapport}
\date{2011-04-26 20:25 CET}

\begin{document}
\maketitle

\subsection{Description}

A t-test report with table of descriptives, diagnostic tests and t-test
specific statistics.

\subsection{Introduction}

In a nutshell, \emph{t-test} is a statistical test that assesses
hypothesis of equality of two means. But in theory, any hypothesis test
that yields statistic which follows
\href{https://en.wikipedia.org/wiki/Student\%27s\_t-distribution}{\emph{t-distribution}}
can be considered a \emph{t-test}. The most common usage of
\emph{t-test} is to:

\begin{itemize}
\item
  compare the mean of a variable with given test mean value -
  \textbf{one-sample \emph{t-test}}
\item
  compare means of two variables from independent samples -
  \textbf{independent samples \emph{t-test}}
\item
  compare means of two variables from dependent samples -
  \textbf{paired-samples \emph{t-test}}
\end{itemize}
\subsection{Overview}

Independent samples \emph{t-test} is carried out with \emph{Internet
usage in leisure time (hours per day)} as dependent variable, and
\emph{Gender} as independent variable. Confidence interval is set to
95\%. Equality of variances wasn't assumed.

\subsection{Descriptives}

In order to get more insight on the underlying data, a table of basic
descriptive statistics is displayed below.

\ctable[pos = H, center, botcap]{llllllllll}
{% notes
}
{% rows
\FL
\textbf{Gender} & \textbf{min} & \textbf{max} & \textbf{mean} & \textbf{sd} & \textbf{var} & \textbf{median} & \textbf{IQR} & \textbf{skewness} & \textbf{kurtosis}
\ML
male & 0 & 12 & 3.2699 & 1.9535 & 3.8161 & 3 & 3 & 0.9479 & 4.0064
\\\noalign{\medskip}
female & 0 & 12 & 3.0643 & 2.3546 & 5.5442 & 2 & 3 & 1.4064 & 4.9089
\LL
}

\subsection{Diagnostics}

Since \emph{t-test} is a parametric technique, it sets some basic
assumptions on distribution shape: it has to be \emph{normal} (or
appoximately normal). A few normality test are to be applied, in order
to screen possible departures from normality.

\subsubsection{Normality Tests}

We will use \emph{Shapiro-Wilk}, \emph{Lilliefors} and
\emph{Anderson-Darling} tests to screen departures from normality in the
response variable (\emph{Internet usage in leisure time (hours per
day)}).

\ctable[pos = H, center, botcap]{lll}
{% notes
}
{% rows
\FL
 & \textbf{N} & \textbf{p}
\ML
Shapiro-Wilk normality test & 0.9001 & 0
\\\noalign{\medskip}
Lilliefors (Kolmogorov-Smirnov) normality test & 0.168 & 0
\\\noalign{\medskip}
Anderson-Darling normality test & 18.753 & 0
\LL
}

As you can see, applied tests confirm departures from normality.

\subsection{Results}

Welch Two Sample t-test was applied, and significant differences were
found.

\ctable[pos = H, center, botcap]{llllll}
{% notes
}
{% rows
\FL
 & \textbf{statistic} & \textbf{df} & \textbf{p} & \textbf{CI(lower)} & \textbf{CI(upper)}
\ML
t & 1.1483 & 457.8625 & 0.2514 & -0.1463 & 0.5576
\LL
}

\subsection{Description}

A t-test report with table of descriptives, diagnostic tests and t-test
specific statistics.

\subsection{Introduction}

In a nutshell, \emph{t-test} is a statistical test that assesses
hypothesis of equality of two means. But in theory, any hypothesis test
that yields statistic which follows
\href{https://en.wikipedia.org/wiki/Student\%27s\_t-distribution}{\emph{t-distribution}}
can be considered a \emph{t-test}. The most common usage of
\emph{t-test} is to:

\begin{itemize}
\item
  compare the mean of a variable with given test mean value -
  \textbf{one-sample \emph{t-test}}
\item
  compare means of two variables from independent samples -
  \textbf{independent samples \emph{t-test}}
\item
  compare means of two variables from dependent samples -
  \textbf{paired-samples \emph{t-test}}
\end{itemize}
\subsection{Overview}

One-sample \emph{t-test} is carried out with \emph{Internet usage in
leisure time (hours per day)} as dependent variable. Confidence interval
is set to 95\%. Equality of variances wasn't assumed.

\subsection{Descriptives}

In order to get more insight on the underlying data, a table of basic
descriptive statistics is displayed below.

\ctable[pos = H, center, botcap]{llllllllll}
{% notes
}
{% rows
\FL
\textbf{Variable} & \textbf{NA} & \textbf{NA} & \textbf{NA} & \textbf{NA} & \textbf{NA} & \textbf{NA} & \textbf{NA} & \textbf{NA} & \textbf{NA}
\ML
Internet usage in leisure time (hours per
day) & 0 & 12 & 3.1994 & 2.1436 & 4.5951 & 3 & 2 & 1.1873 & 4.547
\LL
}

\subsection{Diagnostics}

Since \emph{t-test} is a parametric technique, it sets some basic
assumptions on distribution shape: it has to be \emph{normal} (or
appoximately normal). A few normality test are to be applied, in order
to screen possible departures from normality.

\subsubsection{Normality Tests}

We will use \emph{Shapiro-Wilk}, \emph{Lilliefors} and
\emph{Anderson-Darling} tests to screen departures from normality in the
response variable (\emph{Internet usage in leisure time (hours per
day)}).

\ctable[pos = H, center, botcap]{lll}
{% notes
}
{% rows
\FL
 & \textbf{N} & \textbf{p}
\ML
Shapiro-Wilk normality test & 0.9001 & 0
\\\noalign{\medskip}
Lilliefors (Kolmogorov-Smirnov) normality test & 0.168 & 0
\\\noalign{\medskip}
Anderson-Darling normality test & 18.753 & 0
\LL
}

As you can see, applied tests confirm departures from normality.

\subsection{Results}

One Sample t-test was applied, and significant differences were found.

\ctable[pos = H, center, botcap]{llllll}
{% notes
}
{% rows
\FL
 & \textbf{statistic} & \textbf{df} & \textbf{p} & \textbf{CI(lower)} & \textbf{CI(upper)}
\ML
t & -0.0072 & 671 & 0.9943 & 3.037 & 3.3618
\LL
}

\begin{center}\rule{3in}{0.4pt}\end{center}

This report was generated with \href{http://www.r-project.org/}{R}
(2.14.0) and \href{http://al3xa.github.com/rapport/}{rapport} (0.1) in
0.801 sec on x86\_64-unknown-linux-gnu platform.

\begin{figure}[htbp]
\centering
\includegraphics{images/logo.png}
\caption{}
\end{figure}

\end{document}
