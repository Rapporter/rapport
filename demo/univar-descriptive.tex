\documentclass{article}
\usepackage{amssymb,amsmath}
\usepackage{ifxetex,ifluatex}
\ifxetex
  \usepackage{fontspec,xltxtra,xunicode}
  \defaultfontfeatures{Mapping=tex-text,Scale=MatchLowercase}
\else
  \ifluatex
    \usepackage{fontspec}
    \defaultfontfeatures{Mapping=tex-text,Scale=MatchLowercase}
  \else
    \usepackage[utf8]{inputenc}
  \fi
\fi
\usepackage{ctable}
\usepackage{float} % provides the H option for float placement
\usepackage{graphicx}
% We will generate all images so they have a width \maxwidth. This means
% that they will get their normal width if they fit onto the page, but
% are scaled down if they would overflow the margins.
\makeatletter
\def\maxwidth{\ifdim\Gin@nat@width>\linewidth\linewidth
\else\Gin@nat@width\fi}
\makeatother
\let\Oldincludegraphics\includegraphics
\renewcommand{\includegraphics}[1]{\Oldincludegraphics[width=\maxwidth]{#1}}
\ifxetex
  \usepackage[setpagesize=false, % page size defined by xetex
              unicode=false, % unicode breaks when used with xetex
              xetex]{hyperref}
\else
  \usepackage[unicode=true]{hyperref}
\fi
\hypersetup{breaklinks=true, pdfborder={0 0 0}}
\newcommand{\textsubscr}[1]{\ensuremath{_{\scriptsize\textrm{#1}}}}
\setlength{\parindent}{0pt}
\setlength{\parskip}{6pt plus 2pt minus 1pt}
\setlength{\emergencystretch}{3em}  % prevent overfull lines
\setcounter{secnumdepth}{0}

\title{Descriptive statistics}
\author{Rapport package team @ https://github.com/aL3xa/rapport}
\date{2011--04--26 20:25 CET}

\begin{document}
\maketitle

\subsection{Description}

This template will return descriptive statistics of a numerical, or a
frequency table of a categorical variable.

\section{\emph{gender} (``Gender'')}

The dataset has \emph{709} observations with \emph{673} valid values
(missing: \emph{36}) in \emph{gender} (``Gender''). This variable seems
to be a factor.

\subsection{Base statistics}

\ctable[pos = H, center, botcap]{llllll}
{% notes
}
{% rows
\FL
 & \textbf{gender} & \textbf{N} & \textbf{pct} & \textbf{cumul.count} & \textbf{cumul.pct}
\ML
1 & male & 410 & 60.9212 & 410 & 60.9212
\\\noalign{\medskip}
2 & female & 263 & 39.0788 & 673 & 100
\\\noalign{\medskip}
Total &  & 673 & 100 & 673 & 100
\LL
}

\subsection{Barplot}

\begin{figure}[htbp]
\centering
\includegraphics{3ed92ab3ffc6e875335e7e8c774c35a8.png}
\caption{}
\end{figure}

It seems that the highest value is \emph{2} which is exactly 2 times
higher than the smallest value (\emph{1}).

\subsection{Description}

This template will return descriptive statistics of a numerical, or a
frequency table of a categorical variable.

\section{\emph{age} (``Age'')}

The dataset has \emph{709} observations with \emph{677} valid values
(missing: \emph{32}) in \emph{age} (``Age''). This variable seems to be
numeric.

\subsection{Base statistics}

\ctable[pos = H, center, botcap]{llll}
{% notes
}
{% rows
\FL
\textbf{value} & \textbf{mean(age)} & \textbf{sd(age)} & \textbf{var(age)}
\ML
(all) & 24.5731 & 6.8491 & 46.9107
\LL
}

\subsection{Histogram}

\begin{figure}[htbp]
\centering
\includegraphics{ac5d789145bdef09b10219ef16429f53.png}
\caption{}
\end{figure}

It seems that the highest value is \emph{58} which is exactly 3.625
times higher than the smallest value (\emph{16}).

The standard deviation is 6.8491 (variance: 46.9107). The expected value
is around 24.5731, somewhere between 24.0572 and 25.0891 (SE: 0.2632).

If we suppose that \emph{Age} is not near to a normal distribution
(test: see below, skewness: 1.9296, kurtosis: 7.4851), checking the
median (23) might be a better option instead of the mean. The
interquartile range (6) measures the statistics dispersion of the
variable (similar to standard deviation) based on median.

\subsection{Normality tests}

\subsection{Introduction}

In statistics, \emph{normality} refers to an assumption that the
distribution of a random variable follows \emph{normal}
(\emph{Gaussian}) distribution. Because of its bell-like shape, it's
also known as the \emph{``bell curve''}. The formula for \emph{normal
distribution} is:

\[f(x) = \frac{1}{\sqrt{2\pi{}\sigma{}^2}} e^{-\frac{(x-\mu{})^2}{2\sigma{}^2}}\]

\emph{Normal distribution} belongs to a \emph{location-scale family} of
distributions, as it's defined two parameters:

\begin{itemize}
\item
  \emph{μ} - \emph{mean} or \emph{expectation} (location parameter)
\item
  \emph{σ\textsuperscript{2}} - \emph{variance} (scale parameter)
\end{itemize}
\begin{figure}[htbp]
\centering
\includegraphics{2f8c434e103f36ec70966b372838d448.png}
\caption{}
\end{figure}

\subsection{Normality Tests}

\subsubsection{Overview}

Various hypothesis tests can be applied in order to test if the
distribution of given random variable violates normality assumption.
These procedures test the H\textsubscr{0} that provided variable's
distribution is \emph{normal}. At this point only few such tests will be
covered: the ones that are available in \texttt{stats} package (which
comes bundled with default R installation) and \texttt{nortest} package
that is
\href{http://cran.r-project.org/web/packages/nortest/index.html}{available}
on CRAN.

\begin{itemize}
\item
  \textbf{Shapiro-Wilk test} is a powerful normality test appropriate
  for small samples. In R, it's implemented in \texttt{shapiro.test}
  function available in \texttt{stats} package.
\item
  \textbf{Lilliefors test} is a modification of \emph{Kolmogorov-Smirnov
  test} appropriate for testing normality when parameters or normal
  distribution (\emph{μ}, \emph{σ\textsuperscript{2}}) are not known.
  \texttt{lillie.test} function is located in \texttt{nortest} package.
\item
  \textbf{Anderson-Darling test} is one of the most powerful normality
  tests as it will detect the most of departures from normality. You can
  find \texttt{ad.test} function in \texttt{nortest} package.
\item
  \textbf{Pearson Χ\textsuperscript{2} test} is another normality test
  which takes more ``traditional'' approach in normality testing.
  \texttt{pearson.test} is located in \texttt{nortest} package.
\end{itemize}
\subsubsection{Results}

Here you can see the results of applied normality tests (\emph{p-values}
less than 0.05 indicate significant discrepancies):

\ctable[pos = H, center, botcap]{lll}
{% notes
}
{% rows
\FL
 & \textbf{H} & \textbf{p}
\ML
shapiro.test & 0.8216 & 0
\\\noalign{\medskip}
lillie.test & 0.17 & 0
\\\noalign{\medskip}
ad.test & 32.1645 & 0
\\\noalign{\medskip}
pearson.test & 625.8479 & 0
\LL
}

So, let's draw some conclusions based on applied normality test:

\begin{itemize}
\item
  according to \emph{Shapiro-Wilk test}, the distribution of \emph{Age}
  is not normal.
\item
  based on \emph{Lilliefors test}, distribution of \emph{Age} is not
  normal
\item
  \emph{Anderson-Darling test} confirms violation of normality
  assumption
\item
  \emph{Pearson's Χ\textsuperscript{2} test} classifies the underlying
  distribution as non-normal
\end{itemize}
\subsection{Diagnostic Plots}

There are various plots that can help you decide about the normality of
the distribution. Only a few most commonly used plots will be shown:
\emph{histogram}, \emph{Q-Q plot} and \emph{kernel density plot}.

\subsubsection{Histogram}

\emph{Histogram} was first introduced by \emph{Karl Pearson} and it's
probably the most popular plot for depicting the probability
distribution of a random variable. However, the decision depends on
number of bins, so it can sometimes be misleading. If the variable
distribution is normal, bins should resemble the ``bell-like'' shape.

\begin{figure}[htbp]
\centering
\includegraphics{ac5d789145bdef09b10219ef16429f53.png}
\caption{}
\end{figure}

\subsubsection{Q-Q Plot}

``Q'' in \emph{Q-Q plot} stands for \emph{quantile}, as this plot
compares empirical and theoretical distribution (in this case,
\emph{normal} distribution) by plotting their quantiles against each
other. For normal distribution, plotted dots should approximate a
``straight'', \texttt{x = y} line.

\begin{figure}[htbp]
\centering
\includegraphics{cbbba756d844aa053998959b73b9feff.png}
\caption{}
\end{figure}

\subsubsection{Kernel Density Plot}

\emph{Kernel density plot} is a plot of smoothed \emph{empirical
distribution function}. As such, it provides good insight about the
shape of the distribution. For normal distributions, it should resemble
the well known ``bell shape''.

\begin{figure}[htbp]
\centering
\includegraphics{2684e7da9f9797bfd75863b18d9d29e9.png}
\caption{}
\end{figure}

\begin{center}\rule{3in}{0.4pt}\end{center}

This report was generated with
\href{http://rapport-package.info/}{rapport}.

\begin{figure}[htbp]
\centering
\includegraphics{images/rapport.png}
\caption{}
\end{figure}

\end{document}
